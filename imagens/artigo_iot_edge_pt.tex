
\documentclass[12pt]{article}

\usepackage[utf8]{inputenc}
\usepackage[T1]{fontenc}
\usepackage[brazil]{babel}
\usepackage{amsmath, amssymb}
\usepackage{hyperref}

\title{A Internet das Coisas e o Papel da Computação na Borda}
\author{Ketley Souza}
\date{\today}

\begin{document}

\maketitle

\begin{otherlanguage*}{brazil}
\begin{center}
\textbf{Resumo}
\end{center}
A Internet das Coisas (IoT) representa uma revolução tecnológica que conecta dispositivos à internet, permitindo a coleta e troca de dados em tempo real. Este artigo explora o conceito de IoT e como a Computação na Borda (Edge Computing) se torna essencial para otimizar desempenho, reduzir latência e melhorar a segurança desses sistemas.
\end{otherlanguage*}

\section{Introdução}
A Internet das Coisas (IoT) vem transformando o modo como interagimos com o mundo digital. Desde casas inteligentes até cidades conectadas, a IoT possibilita a automação e o monitoramento remoto em diversos setores.

No entanto, com o aumento no número de dispositivos conectados, surgem desafios relacionados ao processamento de grandes volumes de dados. Nesse contexto, a Computação na Borda (Edge Computing) surge como uma solução eficaz.

\section{O que é IoT}
IoT é uma rede de dispositivos físicos equipados com sensores, software e outras tecnologias para se conectarem e trocarem dados com outros dispositivos e sistemas pela internet.

Exemplos comuns incluem relógios inteligentes, sensores em fábricas, sistemas de segurança e eletrodomésticos inteligentes.

\section{Computação na Borda}
A Computação na Borda refere-se ao processamento de dados próximo à origem, ou seja, no próprio dispositivo ou em servidores locais, ao invés de enviar tudo para a nuvem.

Essa abordagem reduz a latência, melhora a eficiência e permite decisões em tempo real — fatores cruciais em aplicações críticas como veículos autônomos e saúde.

\section{Benefícios da Integração IoT e Edge}
\begin{itemize}
    \item \textbf{Menor latência}: respostas mais rápidas sem depender da nuvem.
    \item \textbf{Maior privacidade}: dados sensíveis podem ser processados localmente.
    \item \textbf{Economia de banda}: menos dados trafegando pela rede.
    \item \textbf{Escalabilidade}: sistemas mais eficientes e robustos.
\end{itemize}

\section{Conclusão}
A combinação de IoT com Edge Computing representa uma evolução natural no desenvolvimento de sistemas inteligentes e conectados. Ao trazer o processamento de dados para mais perto da fonte, torna-se possível criar soluções mais rápidas, seguras e adaptáveis.

\section*{Referências}
\begin{thebibliography}{9}

\bibitem{iotbook}
GUBBI, Jayavardhana et al. \textit{Internet of Things (IoT): A vision, architectural elements, and future directions}. Future Generation Computer Systems, 2013.

\bibitem{edgebook}
Shi, Weisong et al. \textit{Edge Computing: Vision and Challenges}. IEEE Internet of Things Journal, 2016.

\end{thebibliography}

\end{document}
